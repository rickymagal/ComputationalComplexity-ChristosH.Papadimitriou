\subsection*{Problem 1.4.12}
\addcontentsline{toc}{subsection}{Problem 1.4.12}

\paragraph{Statement.}
Show that, if in the augmentation algorithm of Section 1.2 we always augment by a shortest path, then
(a) the distance of a node from \(s\) cannot decrease from an \(N(f)\) to the next. Furthermore,
(b) if edge \((i,j)\) is the bottleneck at one stage (the edge along the augmenting path with the smallest capacity),
then the distance of \(i\) from \(s\) must increase before it becomes a bottleneck in the opposite direction.
(c) Conclude that there are at most \(O(|E||V|)\) augmentations.


\paragraph{Solution.}
We analyze the augmentation algorithm for \gls{maxflow} (Ford--Fulkerson), under the rule:
at each iteration, choose an \gls{augmentingpath} of minimum number of edges in the current
\gls{residualnetwork} \(N(f)\), and augment by the bottleneck \gls{residualcapacity} along that path.
This is the \gls{edmondskarp} strategy. \cite{edmonds_karp_1972,ford_fulkerson_1962}

For a flow \(f\), let \(\dist_f(v)\) be the length of a shortest directed path from \(s\) to \(v\) in \(N(f)\)
(or \(\infty\) if none exists). Let \(f'\) be the flow obtained after one augmentation.

\begin{enumerate}[label=(\alph*)]
\item \textbf{Distances do not decrease.}
We claim that for every vertex \(v\in V\), \(\dist_{f'}(v)\ge \dist_f(v)\).

Consider a shortest \(s\)-to-\(v\) path \(Q\) in \(N(f')\). Let \((x,y)\) be the last edge of \(Q\), so \(y=v\) and
\(\dist_{f'}(x)=\dist_{f'}(y)-1\). Choose \(v\) with \(\dist_{f'}(v)<\dist_f(v)\) minimizing \(\dist_{f'}(v)\).
Then \(\dist_{f'}(x)\ge \dist_f(x)\) by minimality.

If \((x,y)\) is also an edge of \(N(f)\), then
\[
\dist_f(y)\le \dist_f(x)+1 \le \dist_{f'}(x)+1 = \dist_{f'}(y),
\]
contradicting \(\dist_{f'}(y)<\dist_f(y)\).

Otherwise, \((x,y)\) was not residual under \(f\) but becomes residual under \(f'\). The only new residual edges
created by an augmentation are reverse edges of the augmented path \(P\): thus \((x,y)\) must be of the form
\((x,y)=(j,i)\) where \((i,j)\) lies on \(P\) in the forward direction. Since \(P\) is a shortest augmenting path
in \(N(f)\), we have \(\dist_f(j)=\dist_f(i)+1\). Also \(\dist_{f'}(x)=\dist_{f'}(j)\ge \dist_f(j)\) by minimality,
so
\[
\dist_{f'}(i)=\dist_{f'}(j)+1 \ge \dist_f(j)+1 = \dist_f(i)+2,
\]
and hence \(\dist_{f'}(y)=\dist_{f'}(i)\ge \dist_f(i)+2\). In particular \(\dist_{f'}(y)\ge \dist_f(y)\),
contradiction. Therefore distances cannot decrease.

\item \textbf{Bottleneck reversal forces a distance increase.}
Suppose that at some stage, on the chosen shortest augmenting path \(P\) in \(N(f)\), the edge \((i,j)\) is a
\gls{bottleneckedge}, i.e., it attains the minimum residual capacity along \(P\). Then \((i,j)\) becomes saturated
after augmentation and the reverse residual edge \((j,i)\) appears in \(N(f')\).

Assume that at a later stage, \((j,i)\) becomes a bottleneck edge on some chosen shortest augmenting path in
\(N(g)\), for a later flow \(g\). Since \((j,i)\) lies on a shortest path in \(N(g)\), we have
\(\dist_g(i)=\dist_g(j)+1\).

From the stage when \((i,j)\) was on a shortest path, we know \(\dist_f(j)=\dist_f(i)+1\). By part (a), distances
never decrease across augmentations, so \(\dist_g(j)\ge \dist_f(j)=\dist_f(i)+1\). Therefore
\[
\dist_g(i)=\dist_g(j)+1 \ge (\dist_f(i)+1)+1 = \dist_f(i)+2.
\]
In particular, the distance from \(s\) to \(i\) must increase before the reverse edge \((j,i)\) can be a bottleneck.

\item \textbf{At most \(O(\lvert E\rvert\lvert V\rvert)\) augmentations.}
Each augmentation saturates at least one bottleneck residual edge on the chosen shortest augmenting path.
Fix an original directed edge \((u,v)\in E\). Consider the events where either \((u,v)\) or \((v,u)\) becomes a
bottleneck edge of a chosen shortest augmenting path (these are the only times this original edge contributes a
bottleneck in the residual network).

By part (b), once \((u,v)\) is a bottleneck, before \((v,u)\) can later be a bottleneck, the distance \(\dist(u)\)
must increase by at least \(2\). Distances are always at most \(\lvert V\rvert-1\) when finite, and by part (a) they
never decrease. Hence the number of alternations of bottleneck direction for this pair is \(O(\lvert V\rvert)\), and
therefore each original edge can be a bottleneck at most \(O(\lvert V\rvert)\) times overall.

Since each augmentation saturates at least one bottleneck edge, the total number of augmentations is at most
\(O(\lvert E\rvert\lvert V\rvert)\).
\end{enumerate}
