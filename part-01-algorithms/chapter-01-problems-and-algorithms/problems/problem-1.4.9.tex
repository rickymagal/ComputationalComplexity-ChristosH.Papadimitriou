\subsection*{Problem 1.4.9}
\addcontentsline{toc}{subsection}{Problem 1.4.9}

\paragraph{Solution.}
Let \(p(n)\) be a polynomial and let \(c>0\) be a constant.

\paragraph{General statement.}
Since \(p\) is a polynomial, there exist constants \(a>0\) and \(k\in\N\) such that \(p(n)\le a n^k\) for all
\(n\ge 1\). We want \(2^{cn} > p(n)\), and it suffices to ensure
\[
2^{cn} > a n^k.
\]
Taking \(\logtwo\) of both sides yields the equivalent inequality
\[
cn > \logtwo a + k\logtwo n.
\]
The left-hand side grows linearly in \(n\), while the right-hand side grows on the order of \(\log n\). Hence
there exists \(n_0\) such that for all \(n\ge n_0\) the inequality holds, proving the claim.

\begin{enumerate}[label=(\alph*)]
\item Here \(p(n)=n^2\) and \(c=1\). We need \(2^n > n^2\) for all \(n\ge n_0\).

Direct check gives \(2^4=16=4^2\) and \(2^5=32>25\). For \(n\ge 5\), the function
\[
g(n)=\frac{2^n}{n^2}
\]
is strictly increasing, since
\[
\frac{g(n+1)}{g(n)}=\frac{2^{n+1}/(n+1)^2}{2^n/n^2} = 2\left(\frac{n}{n+1}\right)^2 > 1.
\]
Therefore \(2^n/n^2 \ge 2^5/5^2 > 1\) for all \(n\ge 5\), so \(n_0=5\) works.

\item Here \(p(n)=100n^{100}\) and \(c=\frac{1}{100}\). We need
\[
2^{n/100} > 100 n^{100}.
\]
Taking \(\logtwo\) yields
\[
\frac{n}{100} > \logtwo 100 + 100\logtwo n.
\]
Define
\[
\phi(n)=\frac{n}{100} - \logtwo 100 - 100\logtwo n.
\]
For \(n>0\),
\[
\phi'(n)=\frac{1}{100} - \frac{100}{n\ln 2}.
\]
Thus \(\phi'(n)>0\) for all \(n \ge 15000\), because then
\(\frac{100}{n\ln 2}\le \frac{100}{15000\ln 2} < \frac{1}{100}\).
So \(\phi\) is increasing for \(n\ge 15000\). It suffices to find one \(n\ge 15000\) with \(\phi(n)>0\).

Take \(n_0=200000\). Since \(\logtwo(200000) < 18\) and \(\logtwo 100 < 7\), we have
\[
\logtwo 100 + 100\logtwo(200000) < 7 + 1800 = 1807 < 2000 = \frac{200000}{100}.
\]
Hence \(\phi(200000)>0\), and by monotonicity \(\phi(n)>0\) for all \(n\ge 200000\). Therefore \(n_0=200000\)
works.
\end{enumerate}
