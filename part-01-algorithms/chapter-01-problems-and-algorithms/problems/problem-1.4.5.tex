\
\subsection*{Problem 1.4.5}
\addcontentsline{toc}{subsection}{Problem 1.4.5}

\providecommand{\dist}{\operatorname{dist}}

\paragraph{Statement.}
(a) Show that a graph is bipartite (that is, its nodes can be partitioned into two sets, not necessarily of
equal cardinality, with edges going only from one to the other) if and only if it has no odd-length cycles.
(b) Describe a polynomial algorithm for testing whether a graph is bipartite.

\paragraph{Solution.}
We consider an undirected graph \(G=(V,E)\). A \gls{bipartitegraph} is a graph whose vertices can be
partitioned into two sets \(A,B\) such that every edge has one endpoint in \(A\) and the other in \(B\).

\begin{enumerate}[label=(\alph*)]
\item \textbf{\(G\) is bipartite iff it has no \gls{oddcycle}.}

(\(\Rightarrow\)) Suppose \(G\) is bipartite with partition \(V=A\cup B\). Along any cycle, vertices must
alternate between \(A\) and \(B\) (because every edge crosses the cut \((A,B)\)). Therefore the cycle length is
even. Hence there is no \gls{oddcycle}.

(\(\Leftarrow\)) Suppose \(G\) has no \gls{oddcycle}. Fix a connected component and choose a root \(r\).
Define a \gls{twocoloring} by \gls{bfs} layers: place a vertex \(v\) in \(A\) iff the shortest-path distance
\(\dist(r,v)\) is even, and in \(B\) iff \(\dist(r,v)\) is odd.

We claim that no edge has both endpoints in \(A\) (and similarly for \(B\)). If there were an edge \(\{x,y\}\in E\)
with \(\dist(r,x)\equiv \dist(r,y)\pmod 2\), let \(P_x\) and \(P_y\) be shortest paths from \(r\) to \(x\) and \(r\)
to \(y\). Let \(z\) be their last common vertex (the point where they diverge). Then the subpaths from \(z\) to
\(x\) and from \(z\) to \(y\), together with the edge \(\{x,y\}\), form a cycle of length
\(\dist(z,x)+\dist(z,y)+1\). Since \(\dist(r,x)\equiv \dist(r,y)\pmod 2\), we have
\(\dist(z,x)\equiv \dist(z,y)\pmod 2\), so \(\dist(z,x)+\dist(z,y)+1\) is odd. This produces a \gls{oddcycle},
contradiction. Therefore every edge crosses between \(A\) and \(B\), and the component is bipartite. Doing this
independently for each component yields a bipartition of \(G\).

\item \textbf{Polynomial-time algorithm to test bipartiteness.}
Run \gls{bfs} (or \gls{dfs}) in each connected component while maintaining a \gls{twocoloring}.
Initialize all vertices as uncolored. For each uncolored start vertex \(r\), color it \(0\) and perform \gls{bfs}.
When exploring an edge \(\{u,v\}\):
\begin{itemize}
\item if \(v\) is uncolored, assign it color \(1-\text{color}(u)\) and continue;
\item if \(v\) is already colored and \(\text{color}(v)=\text{color}(u)\), reject (an odd cycle exists).
\end{itemize}
If all edges are processed without conflict, accept and return the induced partition.

With \glspl{adjacencylist}, each edge is examined at most twice (once from each endpoint), so the running time is
\(O(\lvert V\rvert+\lvert E\rvert)\).
\end{enumerate}
