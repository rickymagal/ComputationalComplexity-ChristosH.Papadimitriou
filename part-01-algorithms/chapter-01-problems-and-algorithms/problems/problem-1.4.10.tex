\subsection*{Problem 1.4.10}
\addcontentsline{toc}{subsection}{Problem 1.4.10}

\paragraph{Statement.}
Let \(f(n)\) and \(g(n)\) be any two of the following functions. Determine whether (i) \(f(n)=O(g(n))\);
(ii) \(f(n)=\Omega(g(n))\); or (iii) \(f(n)=\Theta(g(n))\):
(a) \(n^2\); \quad (b) \(n^3\); \quad (c) \(n^2\log n\)
(d) \(2^n\); \quad (e) \(n^n\); \quad (f) \(n^{\log n}\)
(g) \(2^{2^n}\); \quad (h) \(2^{2^{n+1}}\); \quad (j) \(n^2\) if \(n\) is odd, \(2^n\) otherwise.


\paragraph{Solution.}
Let the functions be:
\[
\begin{aligned}
(a)\;& n^2, \qquad
(b)\;& n^3, \qquad
(c)\;& n^2\log n, \qquad
(d)\;& 2^n, \\
(e)\;& n^n, \qquad
(f)\;& n^{10\log_2 n}, \qquad
(g)\;& 2^{2n}, \qquad
(h)\;& 2^{2n+1}, \qquad
(j)\;& \begin{cases}
n^2 & \text{if \(n\) is odd},\\
2^n & \text{if \(n\) is even}.
\end{cases}
\end{aligned}
\]
We use standard \gls{asymptoticO}, \gls{asymptoticOmega}, \gls{asymptoticTheta}. We also write
\(f \in o(g)\) (see \gls{littleo}) to mean \(\lim_{n\to\infty} f(n)/g(n)=0\).

\paragraph{Step 1: A strict growth chain for (a),(b),(c),(d),(e),(f),(g),(h).}
We claim the following asymptotic hierarchy:
\[
n^2 \in o(n^2\log n) \in o(n^3) \in o\!\left(n^{10\log_2 n}\right) \in o(2^n) \in o(2^{2n}),
\quad 2^{2n} \in \Theta(2^{2n+1}),
\quad 2^{2n+1} \in o(n^n).
\]
Each step is justified as follows:
\begin{itemize}
\item \(\dfrac{n^2}{n^2\log n}=\dfrac{1}{\log n}\to 0\).
\item \(\dfrac{n^2\log n}{n^3}=\dfrac{\log n}{n}\to 0\).
\item \(\dfrac{n^3}{n^{10\log_2 n}} = n^{3-10\log_2 n}\to 0\) since \(10\log_2 n \to \infty\).
\item Compare \(n^{10\log_2 n}\) to \(2^n\) by taking \(\log_2\):
\[
\log_2\!\left(n^{10\log_2 n}\right)=10(\log_2 n)^2 = o(n) = \log_2(2^n),
\]
hence \(n^{10\log_2 n} \in o(2^n)\).
\item \(\dfrac{2^n}{2^{2n}} = 2^{-n}\to 0\).
\item \(2^{2n+1}=2\cdot 2^{2n}\), so \(2^{2n}\in\Theta(2^{2n+1})\).
\item Compare \(2^{2n+1}\) to \(n^n\) by taking \(\log_2\):
\[
\log_2(2^{2n+1})=2n+1,\qquad \log_2(n^n)=n\log_2 n,
\]
and \(n\log_2 n \gg 2n\), so \(2^{2n+1}\in o(n^n)\).
\end{itemize}

\paragraph{Consequence (pairs not involving (j)).}
For any two distinct functions among \((a),(b),(c),(d),(e), \\(f),(g),(h)\), the slower-growing one is in
\gls{asymptoticO} of the faster-growing one, and the faster-growing one is in \gls{asymptoticOmega} of the slower.
The only \gls{asymptoticTheta} equivalence among these is \((g)\) and \((h)\), since they differ by a factor \(2\).

\paragraph{Step 2: The exceptional function (j).}
Define
\[
J(n)=\begin{cases}
n^2 & \text{if \(n\) is odd},\\
2^n & \text{if \(n\) is even}.
\end{cases}
\]
Then \(J(n)\ge n^2\) for all \(n\), and \(J(2k)=2^{2k}\).

\begin{itemize}
\item Versus \((a)\) \(n^2\): \(J\in \Omega(n^2)\) (trivial) but \(J\notin O(n^2)\) because on even \(n\),
\(J(n)=2^n\) and \(2^n/n^2\to\infty\). Thus \(J\) and \(n^2\) are not \gls{asymptoticTheta}-equivalent.
Equivalently, \(n^2\in O(J)\) and \(J\in \Omega(n^2)\).

\item Versus \((c)\) \(n^2\log n\), \((b)\) \(n^3\), and \((f)\) \(n^{10\log_2 n}\): \(J\) is \gls{incomparablefunc}
with each of these functions. Indeed, on even \(n\), \(J(n)=2^n\) dominates any of them; hence \(J\notin O(g)\).
On odd \(n\), \(J(n)=n^2\) is dominated by each of them for large \(n\); hence \(J\notin \Omega(g)\).
Therefore neither (i) nor (ii) holds (and thus not (iii)) for these pairs.

\item Versus \((d)\) \(2^n\): \(J\in O(2^n)\) because for even \(n\), \(J(n)=2^n\), and for odd \(n\ge 5\),
\(n^2 \le 2^n\). But \(J\notin \Omega(2^n)\) since on odd \(n\), \(J(n)/2^n = n^2/2^n \to 0\).
So \(J\) is strictly smaller than \(2^n\) along an infinite subsequence. Also \(2^n \notin O(J)\), because on odd
\(n\), \(2^n/J(n)=2^n/n^2\to\infty\).

\item Versus \((g)\) \(2^{2n}\) and \((h)\) \(2^{2n+1}\): \(J\in O(2^{2n})\) (and hence \(O(2^{2n+1})\)), but
\(J\notin \Omega(2^{2n})\) since on odd \(n\), \(J(n)=n^2\) and \(n^2/2^{2n}\to 0\). Likewise,
\(2^{2n}\notin O(J)\) because on even \(n\), \(2^{2n}/J(n)=2^{2n}/2^n=2^n\to\infty\).

\item Versus \((e)\) \(n^n\): \(J\in O(n^n)\) (since \(J(n)\le 2^n \le n^n\) for all \(n\ge 2\)), but
\(J\notin \Omega(n^n)\) because on odd \(n\), \(J(n)=n^2\) and \(n^2/n^n = n^{2-n}\to 0\).
Also \(n^n\notin O(J)\) since on even \(n\), \(n^n/J(n)=n^n/2^n\to\infty\).
\end{itemize}

This fully determines which of (i) \(f\in O(g)\), (ii) \(f\in \Omega(g)\), or (iii) \(f\in \Theta(g)\) holds for
any pair from the list: use the total order in Step 1 for pairs not involving \((j)\), and the case analysis above
for pairs involving \(J\).
