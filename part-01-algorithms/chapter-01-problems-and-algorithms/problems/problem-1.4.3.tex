\subsection*{Problem 1.4.3}
\addcontentsline{toc}{subsection}{Problem 1.4.3}

\paragraph{Statement.}
Show that in the search algorithm each edge of \(G\) is processed at most once.
Conclude that the algorithm can be implemented to run in \(O(|E|)\) steps.

\paragraph{Solution.}
Implement the search using \glspl{adjacencylist}. For each vertex \(u\), store its outgoing neighbors
\(\Gamma^+(u)=\{v : (u,v)\in E\}\). Maintain a boolean array \(\texttt{seen}[\cdot]\) and a \gls{worklist}
(queue for \gls{bfs} or stack for \gls{dfs}).

Initialize \(\texttt{seen}[1]=\texttt{true}\) and push \(1\) into the \gls{worklist}. While the \gls{worklist}
is nonempty, pop a vertex \(u\) and scan its \gls{adjacencylist} \(\Gamma^+(u)\). For each neighbor
\(v\in \Gamma^+(u)\), the edge \((u,v)\) is \emph{processed} by performing the constant-time check of
\(\texttt{seen}[v]\) and, if false, setting it to true and pushing \(v\) into the \gls{worklist}.

\paragraph{Each edge is processed at most once.}
A vertex \(u\) is popped at most once: after \(\texttt{seen}[u]\) becomes true, \(u\) is inserted into the
\gls{worklist} at most once, and therefore its \gls{adjacencylist} is scanned at most once. Since an edge
\((u,v)\) appears only in the \gls{adjacencylist} of its tail \(u\), it is examined only when \(u\) is popped.
Hence each edge \((u,v)\in E\) is processed at most once.

\paragraph{Running time.}
The total cost of scanning all \glspl{adjacencylist} is
\[
\sum_{u\in V} \abs{\Gamma^+(u)} = \abs{E},
\]
and the per-edge work is constant. Therefore the algorithm runs in \(\bigO(\abs{E})\) edge-processing steps
(and in \(\bigO(\abs{V}+\abs{E})\) total time if one also counts initialization and \gls{worklist} operations).
