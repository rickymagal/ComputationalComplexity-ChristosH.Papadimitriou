\subsection*{Problem 1.4.2}
\addcontentsline{toc}{subsection}{Problem 1.4.2}

\paragraph{Statement.}
(a) Show by induction on \(i\) that, if \(v\) is the \(i\)th node added by the search algorithm of Section 1.1 to the set \(S\), then there is a path from node \(1\) to \(v\).
(b) Show by induction on \(\ell\) that if node \(v\) is reachable from node \(1\) via a path with \(\ell\) edges, then the search algorithm will add \(v\) to set \(S\).

\paragraph{Solution.}
Consider the standard reachability search from a start vertex \(1\). Maintain a set \(S\) of discovered
vertices, initialized as \(S=\{1\}\). Repeatedly add a vertex \(v\notin S\) whenever there exists an edge
\((u,v)\in E\) with \(u\in S\). Stop when no such edge exists.

\begin{enumerate}[label=(\alph*)]
\item We prove by induction on \(i\) that if \(v\) is the \(i\)-th vertex added to \(S\), then there exists a
directed path from \(1\) to \(v\).

\emph{Base case.} The first vertex added is \(1\), and the length-\(0\) path suffices.

\emph{Inductive step.} Assume the claim holds for the first \(i-1\) added vertices. Let \(v\) be the \(i\)-th
vertex added. By the rule of the algorithm, \(v\) is added because there exists an edge \((u,v)\in E\) with
\(u\in S\) at that moment. Since \(u\) was added earlier, by the induction hypothesis there is a path from
\(1\) to \(u\). Appending the edge \((u,v)\) yields a path from \(1\) to \(v\).

\item We prove by induction on \(\ell\) that if \(v\) is reachable from \(1\) by a directed path of \(\ell\)
edges, then the algorithm will add \(v\) to \(S\).

\emph{Base case.} If \(\ell=0\), then \(v=1\), and \(1\in S\) initially.

\emph{Inductive step.} Assume the statement holds for all vertices reachable by a path of length \(\ell-1\),
for some \(\ell\ge 1\). Let \(v\) be reachable from \(1\) by a path of length \(\ell\), and fix such a path.
Let \(u\) be the vertex immediately preceding \(v\) on this path. Then \(u\) is reachable by a path of length
\(\ell-1\), hence \(u\) is added to \(S\) by the induction hypothesis. Once \(u\in S\), the edge \((u,v)\)
witnesses that \(v\) is eligible to be added, so the algorithm will eventually add \(v\) (it continues until no
eligible edge remains).
\end{enumerate}

Parts (a) and (b) together imply that the algorithm adds exactly the vertices reachable from \(1\).
