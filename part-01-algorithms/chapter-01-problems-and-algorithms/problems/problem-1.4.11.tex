\
\subsection*{Problem 1.4.11}
\addcontentsline{toc}{subsection}{Problem 1.4.11}

\paragraph{Statement.}
The max-flow min-cut theorem: In any network the value of the maximum flow equals the capacity of the minimum cut.

Here by a cut in a network \(N=(V,E,s,t,c)\) we mean a set \(S\) of nodes such that \(s\in S\) and \(t\notin S\).
The capacity of the cut \(S\) is the sum of the capacities of all edges going out of \(S\).

Give a proof of the max-flow min-cut theorem based on the augmentation algorithm of Section 1.2.
(The value of any flow is always at most as large as the capacity of any cut. And, when the algorithm terminates,
it suggests a cut which is at most as large as the flow.)

\paragraph{Solution.}
Let \(N=(V,E,s,t,c)\) be a directed network with capacities \(c:E\to\mathbb{R}_{\ge 0}\).
A (feasible) flow is a function \(f:E\to\mathbb{R}_{\ge 0}\) such that \(0\le f(e)\le c(e)\) for all \(e\in E\), and for every
vertex \(v\in V\setminus\{s,t\}\) we have flow conservation:
\begin{equation}\label{eq:conservation}
\sum_{(u,v)\in E} f(u,v) \;=\; \sum_{(v,w)\in E} f(v,w).
\end{equation}
The value of \(f\) is
\begin{equation}\label{eq:value}
\lvert f\rvert \;=\; \sum_{(s,w)\in E} f(s,w) \;-\; \sum_{(u,s)\in E} f(u,s).
\end{equation}

\paragraph{1) Any flow is bounded by any cut.}
Fix a cut \(S\subseteq V\) with \(s\in S\) and \(t\notin S\). Its capacity is
\begin{equation}\label{eq:cutcap}
c(S) \;=\; \sum_{\substack{(u,v)\in E\\ u\in S,\,v\notin S}} c(u,v).
\end{equation}
Define the net flow leaving \(S\) under \(f\) by
\begin{equation}\label{eq:netout}
f(S) \;=\; \sum_{\substack{(u,v)\in E\\ u\in S,\,v\notin S}} f(u,v)\;-\;\sum_{\substack{(u,v)\in E\\ u\notin S,\,v\in S}} f(u,v).
\end{equation}
By summing the conservation equations \eqref{eq:conservation} over all \(v\in S\setminus\{s\}\), all internal edges cancel, and one obtains
\(f(S)=\lvert f\rvert\). Therefore,
\begin{equation}\label{eq:flowlecut}
\lvert f\rvert \;=\; f(S)
\;\le\; \sum_{\substack{(u,v)\in E\\ u\in S,\,v\notin S}} f(u,v)
\;\le\; \sum_{\substack{(u,v)\in E\\ u\in S,\,v\notin S}} c(u,v)
\;=\; c(S).
\end{equation}
So the value of any flow is at most the capacity of any cut.

\paragraph{2) The augmentation algorithm produces a cut matching the final flow.}
Consider the augmentation (Ford--Fulkerson) algorithm of Section 1.2.
Given a current flow \(f\), define the residual capacity of a forward edge \((u,v)\in E\) as \(c(u,v)-f(u,v)\), and allow a
backward residual edge \((v,u)\) with residual capacity \(f(u,v)\).
Let \(N(f)\) be the resulting residual network.

The algorithm repeats: find an \(s\)-to-\(t\) path in \(N(f)\) with all residual capacities positive (an augmenting path),
augment by the minimum residual capacity along that path, and update \(f\).
It terminates exactly when there is \emph{no} augmenting path from \(s\) to \(t\) in \(N(f)\).

Let \(f^\star\) be the flow when the algorithm terminates. In the residual network \(N(f^\star)\), let
\begin{equation}\label{eq:reachable}
S \;=\; \{\,v\in V : \text{there is a path from } s \text{ to } v \text{ in } N(f^\star)\,\}.
\end{equation}
By definition, \(s\in S\). Since there is no augmenting path to \(t\), we have \(t\notin S\). Thus \(S\) is a cut.

\paragraph{Claim: \(c(S)=\lvert f^\star\rvert\).}
We show two properties of edges crossing the cut.

\begin{itemize}
\item If \((u,v)\in E\) with \(u\in S\) and \(v\notin S\), then its forward residual capacity is \(0\); otherwise
there would be a residual edge \((u,v)\) of positive capacity and \(v\) would be reachable, contradicting \(v\notin S\).
Hence \(c(u,v)-f^\star(u,v)=0\), i.e., \(f^\star(u,v)=c(u,v)\). Such edges are \emph{saturated}.

\item If \((u,v)\in E\) with \(u\notin S\) and \(v\in S\), then its backward residual capacity must be \(0\); otherwise
there would be a residual edge \((v,u)\) of positive capacity (coming from the possibility of canceling flow on \((u,v)\)),
and then \(u\) would be reachable from \(s\), contradicting \(u\notin S\). Therefore \(f^\star(u,v)=0\). Such edges carry no flow.
\end{itemize}

Using these facts in \eqref{eq:netout} for the cut \(S\),
\begin{align*}
\lvert f^\star\rvert
&= f^\star(S)\\
&= \sum_{\substack{(u,v)\in E\\ u\in S,\,v\notin S}} f^\star(u,v)\;-\;\sum_{\substack{(u,v)\in E\\ u\notin S,\,v\in S}} f^\star(u,v)\\
&= \sum_{\substack{(u,v)\in E\\ u\in S,\,v\notin S}} c(u,v)\;-\;0
\;=\; c(S).
\end{align*}

\paragraph{3) Conclusion (max-flow min-cut).}
From Part 1, any flow value is at most any cut capacity. From Part 2, the terminating flow \(f^\star\) satisfies
\(\lvert f^\star\rvert=c(S)\) for a particular cut \(S\). Therefore \(\lvert f^\star\rvert\) is the maximum possible flow value
and \(c(S)\) is the minimum possible cut capacity, proving the max-flow min-cut theorem.
