\subsection*{Problem 1.4.4}
\addcontentsline{toc}{subsection}{Problem 1.4.4}

\paragraph{Statement.}
(a) A directed graph is acyclic if it has no cycles. Show that any acyclic graph has a source
(a node with no incoming edges).
(b) Show that a graph with \(n\) nodes is acyclic if and only if its nodes can be numbered
\(1\) to \(n\) so that all edges go from lower to higher numbers (use the property in (a) above repeatedly).
(c) Describe a polynomial-time algorithm that decides whether a graph is acyclic.
(Implement the idea in (b) above. With care, your algorithm should not spend more than a constant amount of time per edge.)


\paragraph{Solution.}
Let \(G=(V,E)\) be a finite directed graph. Recall that \(G\) is \gls{dag} if it has no directed cycle.

\begin{enumerate}[label=(\alph*)]
\item \textbf{Every \gls{dag} has a \gls{source}.}
Assume \(G\) is acyclic. We show that \(G\) has a vertex of \gls{indegree} \(0\).

Suppose, for contradiction, that every vertex has \gls{indegree} at least \(1\). Pick any \(v_0\in V\).
Since \(v_0\) has an incoming edge, choose \(v_1\) with \((v_1,v_0)\in E\). Similarly, choose
\(v_2\) with \((v_2,v_1)\in E\), and continue, obtaining a sequence \(v_0,v_1,v_2,\dots\).
Because \(V\) is finite, some vertex repeats: \(v_i=v_j\) for some \(i<j\). Then
\((v_{i+1},v_i),(v_{i+2},v_{i+1}),\dots,(v_j,v_{j-1})\) forms a directed cycle, contradicting acyclicity.
Hence some vertex must have \gls{indegree} \(0\), i.e., a \gls{source} exists.

\item \textbf{Acyclicity iff a \gls{topologicalorder} exists.}

(\(\Rightarrow\)) Assume \(G\) is acyclic. Repeatedly select a \gls{source} (guaranteed by (a)), assign it the
next label \(1,2,\dots\), and delete it together with its outgoing edges. Deleting a vertex cannot create a
cycle, so the remaining graph stays acyclic and the process continues until all vertices are labeled.
If a directed edge \((u,v)\) remains when \(u\) is labeled, then \(v\) is still present; thus \(v\) is labeled later.
Therefore every edge goes from a lower number to a higher number, i.e., the labeling is a \gls{topologicalorder}.

(\(\Leftarrow\)) Conversely, assume the vertices can be numbered \(1,\dots,n\) so that every edge \((u,v)\in E\)
satisfies \(\text{num}(u) < \text{num}(v)\). Along any directed walk, the numbers strictly increase, so no directed
cycle can exist. Hence \(G\) is acyclic.

\item \textbf{Algorithm in \(O(\lvert E\rvert+\lvert V\rvert)\) time, with constant work per edge.}
Implement the idea in (b) via \gls{kahnsalgorithm}.

\begin{enumerate}[label=\arabic*.]
\item Compute \(\text{indeg}[v]\) for all \(v\in V\) by scanning the edge list once and incrementing
\(\text{indeg}[v]\) for each edge \((u,v)\in E\).
\item Initialize a queue (a \gls{worklist}) with all vertices \(v\) such that \(\text{indeg}[v]=0\).
\item Repeatedly pop a vertex \(u\) from the queue, output it next in the ordering, and for each outgoing edge
\((u,w)\), decrement \(\text{indeg}[w]\). If \(\text{indeg}[w]\) becomes \(0\), push \(w\) into the queue.
\end{enumerate}

If the algorithm outputs all vertices, then it has produced a \gls{topologicalorder}, so \(G\) is acyclic by (b).
If it stops early (queue empty while some vertices remain), then the remaining subgraph has no \gls{source},
so by (a) it must contain a directed cycle; hence \(G\) is not acyclic.

Each edge \((u,w)\) is examined exactly once when \(u\) is popped, and causes one constant-time decrement, so the
total edge-processing work is \(O(\lvert E\rvert)\). The remaining work (initialization, queue operations) is
\(O(\lvert V\rvert)\). Thus the algorithm runs in \(O(\lvert E\rvert+\lvert V\rvert)\) time.
\end{enumerate}
