% problem-2.8.14.tex
\subsection*{Problem 2.8.14}
\addcontentsline{toc}{subsection}{Problem 2.8.14}

\paragraph{Statement.}
Give a detailed proof of Theorem 2.3.
That is, give an explicit mathematical construction of the simulating machine in terms of the simulated machine
(assume the latter has one string, besides the input string).

\paragraph{Solution.}
We recall the theorem (as stated in the book \cite{papadimitriou}):

\begin{quote}
\textbf{Theorem 2.3.} Let \(L\) be a language in \(\mathrm{SPACE}(f(n))\). Then, for any \(\varepsilon>0\),
\(L \in \mathrm{SPACE}(2+\varepsilon f(n))\).
\end{quote}

We give an explicit \gls{spacecompression} construction for the case in which the original decider has a read-only input string and a single work string.

\subsubsection*{Model}
Let \(M=(Q,\Sigma,\Gamma,\delta,s,\mathrm{yes},\mathrm{no})\) be a deterministic \gls{turingmachine} with:
\begin{itemize}
\item a read-only input tape over \(\Sigma\), and
\item one read/write work tape over \(\Gamma\), with a distinguished blank symbol in \(\Gamma\).
\end{itemize}
Let \(f:\mathbb{N}\to\mathbb{N}\). Assume that on every input \(x\) of length \(n\), during its computation \(M\) scans at most \(f(n)\) distinct work-tape cells.

Fix \(\varepsilon>0\). We construct a machine \(M_{\varepsilon}\) that decides the same language as \(M\) and, on all inputs of length \(n\), scans at most \(2+\varepsilon f(n)\) work-tape cells.

\subsubsection*{Construction: pack blocks of work symbols}
Let \(b=\left\lceil \dfrac{1}{\varepsilon}\right\rceil\). Then \(1/b \le \varepsilon\).

The idea is to store \(b\) consecutive work-tape cells of \(M\) inside a single work-tape cell of \(M_{\varepsilon}\).
Define the packed alphabet \(\Gamma'=\Gamma^b\).
A symbol of \(\Gamma'\) is a \(b\)-tuple \(T=(T[0],\dots,T[b-1])\) of symbols in \(\Gamma\).

\paragraph{Work-tape encoding.}
Fix an input \(x\) of length \(n\).
At any time, let the work tape of \(M\) (restricted to the \(f(n)\) scanned cells) be the string
\[
w = w_0 w_1 \cdots w_{f(n)-1}\in \Gamma^{f(n)}.
\]
We encode \(w\) as a string \(W\in(\Gamma')^*\) on the single work tape of \(M_{\varepsilon}\) by grouping into blocks:
for each block index \(j\ge 0\),
\[
W[j] \;=\; \bigl(w_{jb+0},\; w_{jb+1},\; \dots,\; w_{jb+(b-1)}\bigr),
\]
where missing positions beyond \(f(n)-1\) are padded with blanks.
Thus \(\lvert W\rvert = \left\lceil \dfrac{f(n)}{b}\right\rceil\) packed cells suffice.

\paragraph{Head position encoding.}
If \(M\)'s work head is currently at position \(i\in\{0,\dots,f(n)-1\}\), write \(i=jb+r\) with \(0\le r<b\).
Then \(M_{\varepsilon}\)'s work head points to packed cell \(j\), and the offset \(r\) is stored in the finite control.
Concretely, the state set of \(M_{\varepsilon}\) is \(Q\times\{0,1,\dots,b-1\}\), plus accepting/rejecting states.

\subsubsection*{Explicit transition function of \(M_{\varepsilon}\)}
Let the current state of \(M_{\varepsilon}\) be \((q,r)\), where \(q\in Q\) is the simulated state of \(M\) and \(r\) is the offset.
Let the current input symbol be \(a\in\Sigma\cup\{\text{blank}\}\).
Let the current packed work symbol be \(T=(T[0],\dots,T[b-1])\in\Gamma'\), so the simulated work symbol is \(T[r]\).

Assume \(\delta\) has the form
\[
\delta(q,a,T[r]) \;=\; (q',a',\gamma',D_{\mathrm{in}},D_{\mathrm{wk}}),
\]
where \(D_{\mathrm{in}},D_{\mathrm{wk}}\in\{\leftarrow,\rightarrow,\text{stay}\}\) are the input/work head moves.

Then \(M_{\varepsilon}\) performs:
\begin{enumerate}
\item \textbf{Write on the packed work tape.}
Let
\[
T'=(T[0],\dots,T[r-1],\gamma',T[r+1],\dots,T[b-1]).
\]
Write \(T'\) in the current packed work cell.
\item \textbf{Move the input head.} Move according to \(D_{\mathrm{in}}\).
\item \textbf{Update \((q,r)\) and move the packed work head.}
\begin{itemize}
\item If \(D_{\mathrm{wk}}=\text{stay}\): keep \(r\) and do not move the packed work head.
\item If \(D_{\mathrm{wk}}=\rightarrow\):
  \begin{itemize}
  \item if \(r<b-1\), set \(r:=r+1\) (packed head stays);
  \item if \(r=b-1\), set \(r:=0\) and move the packed head right by one cell.
  \end{itemize}
\item If \(D_{\mathrm{wk}}=\leftarrow\):
  \begin{itemize}
  \item if \(r>0\), set \(r:=r-1\) (packed head stays);
  \item if \(r=0\), set \(r:=b-1\) and move the packed head left by one cell.
  \end{itemize}
\end{itemize}
\item \textbf{Update the simulated control state.} Set \(q:=q'\).
Map \(q'=\mathrm{yes}\) and \(q'=\mathrm{no}\) to the corresponding halting states of \(M_{\varepsilon}\).
\end{enumerate}

This defines the transition function of \(M_{\varepsilon}\) purely in terms of \(\delta\).
Correctness follows by induction on the number of simulated steps: the packed tape and the offset encode exactly the work tape and work-head position of \(M\), and the update rule applies exactly the same local rewrite and head movement as \(\delta\).

\subsubsection*{Space bound}
On inputs of length \(n\), \(M\) scans at most \(f(n)\) work cells.
The encoding uses at most \(\left\lceil f(n)/b\right\rceil\) packed cells.
Reserve a constant number of extra packed cells for bookkeeping; absorb this into \(2\).

Using \(\lceil u\rceil \le u+1\) and \(1/b\le \varepsilon\),
\[
\left\lceil \frac{f(n)}{b}\right\rceil \;\le\; \frac{f(n)}{b} + 1 \;\le\; \varepsilon f(n) + 1.
\]
Hence \(M_{\varepsilon}\) scans at most \(2+\varepsilon f(n)\) packed work cells on inputs of length \(n\), proving Theorem 2.3 in the one-work-tape case.

\subsubsection*{Remark}
If the original machine has a constant number of work tapes, one can first combine them into one tape using tracks (a constant-factor blowup) and then apply the same packing argument, preserving the statement \( \mathrm{SPACE}(f(n)) \subseteq \mathrm{SPACE}(2+\varepsilon f(n))\).
