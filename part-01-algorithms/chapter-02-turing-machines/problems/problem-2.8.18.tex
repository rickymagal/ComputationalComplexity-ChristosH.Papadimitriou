% problem-2.8.18.tex
\subsection*{Problem 2.8.18}
\addcontentsline{toc}{subsection}{Problem 2.8.18}

\paragraph{Statement.}
\begin{enumerate}[label=(\alph*)]
\item Show that a nondeterministic one-way finite automaton (recall 2.8.11) can be simulated by a deterministic one, with possibly exponentially many states.
(Construct a deterministic automaton each state of which is a set of states of the given nondeterministic one.)
\item Show that the exponential growth in (a) is inherent.
(Consider the language \(L=\{x\sigma : x \in (\Sigma\setminus\{\sigma\})^*\}\), where the alphabet \(\Sigma\) has \(n\) symbols.)
\end{enumerate}

\paragraph{Solution.}
\begin{enumerate}[label=(\alph*)]
\item \textbf{Subset construction (\gls{nfa} \(\to\) \gls{dfa}).}
Let \(N=(Q,\Sigma,\delta,q_0,F)\) be a (one-way) \gls{nfa}. Here \(\delta(q,a)\subseteq Q\) is the set of possible next states from \(q\) on input symbol \(a\).
(If \(\varepsilon\)-moves are allowed, replace the start set below by the \(\varepsilon\)-closure of \(q_0\).)

Define a \gls{dfa}
\(
D=(Q',\Sigma,\delta',q_0',F')
\)
by:
\begin{itemize}
\item \(Q' = 2^Q\) (the set of all subsets of \(Q\));
\item \(q_0' = \{q_0\}\);
\item for \(S\subseteq Q\) and \(a\in\Sigma\),
\begin{equation}\label{eq:subset}
\delta'(S,a)=\bigcup_{q\in S}\delta(q,a);
\end{equation}
\item \(F'=\{S\subseteq Q : S\cap F\neq\emptyset\}\).
\end{itemize}

Let \(\hat{\delta}\) and \(\hat{\delta}'\) be the extended transition functions of \(N\) and \(D\), respectively.
We prove by induction on \(|w|\) that for every \(w\in\Sigma^*\),
\begin{equation}\label{eq:reachset}
\hat{\delta}'(\{q_0\},w)=\{\,q\in Q : q \text{ is reachable in }N\text{ from }q_0\text{ by reading }w\,\}.
\end{equation}

\emph{Base case.} For \(w=\epsilon\), \(\hat{\delta}'(\{q_0\},\epsilon)=\{q_0\}\), which matches the right-hand side.

\emph{Inductive step.} Write \(w=ua\) with \(a\in\Sigma\).
Assuming \eqref{eq:reachset} holds for \(u\), we have
\[
\hat{\delta}'(\{q_0\},ua)=\delta'(\hat{\delta}'(\{q_0\},u),a)
=\bigcup_{q\in \hat{\delta}'(\{q_0\},u)}\delta(q,a)
\]
by \eqref{eq:subset}.
By the induction hypothesis, \(\hat{\delta}'(\{q_0\},u)\) is exactly the set of states reachable in \(N\) after reading \(u\), so the union above is exactly the set of states reachable after reading \(ua\).
Thus \eqref{eq:reachset} holds for \(ua\) as well.

Therefore \(D\) accepts \(w\) iff \(\hat{\delta}'(\{q_0\},w)\cap F\neq\emptyset\), which is equivalent to \(N\) having at least one accepting path on \(w\).
Hence \(L(D)=L(N)\), and \(|Q'|\le 2^{|Q|}\).
This is the classical \gls{subsetconstruction} (see \cite{hopcroftullman1979,sipser2013}).

\item \textbf{The exponential blow-up can be necessary.}
To avoid ambiguity in the hint's notation, interpret the language as
\begin{equation}\label{eq:Ldef}
L \;=\; \{\,x\sigma : \sigma\in\Sigma,\ x\in(\Sigma\setminus\{\sigma\})^*\,\},
\end{equation}
i.e., \(w\in L\) iff the last symbol of \(w\) does not appear earlier in \(w\).
Assume \(|\Sigma|=n\).

\paragraph{An \gls{nfa} of size \(O(n)\) for \(L\).}
Build an \gls{nfa} \(N\) that nondeterministically guesses the final symbol \(\sigma\).
Use states:
one start state \(q_{\mathrm{start}}\), one accepting state \(q_{\mathrm{acc}}\), and for each \(\sigma\in\Sigma\) a state \(q_\sigma\).
From \(q_{\mathrm{start}}\), add an \(\varepsilon\)-transition to every \(q_\sigma\).
In state \(q_\sigma\), on any input symbol \(a\neq \sigma\) stay in \(q_\sigma\); on input \(\sigma\) go to \(q_{\mathrm{acc}}\).
From \(q_{\mathrm{acc}}\), on any further input symbol move to a dead sink; acceptance occurs only if the input ends immediately after reading \(\sigma\).
This \gls{nfa} has \(n+O(1)\) states and recognizes \(L\).

\paragraph{Any equivalent \gls{dfa} needs at least \(2^n\) states.}
We use the \gls{myhillnerode} theorem (Problem 2.8.11(c)).
For each subset \(A\subseteq\Sigma\), choose a string \(x_A\) that lists exactly the symbols of \(A\) once each (in any fixed order), and no other symbols.
We claim that if \(A\neq B\) then \(x_A \not\equiv_L x_B\).

Pick \(a\in A\triangle B\) and assume \(a\in A\setminus B\).
Let \(z=a\). Then:
\begin{itemize}
\item \(x_A a \notin L\), because the last symbol \(a\) already appears earlier in \(x_A\);
\item \(x_B a \in L\), because \(a\notin B\) implies \(x_B\in(\Sigma\setminus\{a\})^*\), hence \(x_B a\in(\Sigma\setminus\{a\})^* a \subseteq L\).
\end{itemize}
Thus \(x_A\) and \(x_B\) are distinguished by the continuation \(z=a\), so they lie in different \(\equiv_L\)-classes.
Therefore \(\equiv_L\) has at least \(2^n\) equivalence classes, one per subset \(A\subseteq\Sigma\), and any \gls{dfa} for \(L\) needs at least \(2^n\) states.

Combining the \(O(n)\)-state \gls{nfa} with the \(2^n\) \gls{dfa} lower bound shows that the exponential blow-up in part (a) is inherent in general.
\end{enumerate}
