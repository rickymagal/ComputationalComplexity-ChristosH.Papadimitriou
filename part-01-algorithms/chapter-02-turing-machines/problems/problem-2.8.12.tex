\subsection*{Problem 2.8.12}
\addcontentsline{toc}{subsection}{Problem 2.8.12}

\paragraph{Statement.}
Show that if a \gls{turingmachine} uses space that is smaller than \(c\log\log n\) for all \(n>0\), then it uses constant space.
(This is from \cite{hartmanis1965hierarchies}.)

Consider a ``state'' of a machine with input to also include the string contents of the work strings.
Then the behavior of the machine on an input prefix can be characterized as a mapping from states to states.
Consider now the shortest input that requires space \(S>0\); all its prefixes must exhibit different behaviors---otherwise a shorter input requires space \(S\).
But the number of behaviors is doubly exponential in \(S\).
Notice that in view of this result, language \(L\) in (g) of Problem 2.8.11 above requires \(O(\log\log n)\) space.

\paragraph{Solution.}
Fix a deterministic \gls{turingmachine} \(M\) in the usual space model (read-only input tape plus a constant number of work tapes).
Let \(s(n)\) be the maximum number of work-tape cells \(M\) ever scans on any input of length \(n\). Assume
\(
s(n) < c \log\log n
\)
for all \(n>0\), where \(c>0\) is a constant to be chosen later.
We prove that \(s(n)\) is bounded by an absolute constant (depending only on \(M\) and \(c\)), i.e., \(M\) uses constant space.

\subsubsection*{1) Extended states and prefix behaviors}
For an integer \(S\ge 0\), define an \gls{extendedstate} of \(M\) at space bound \(S\) to consist of:
\begin{itemize}
\item the control state \(q\in Q\),
\item the contents of all work tapes restricted to the first \(S\) visited cells (an element of \(\Gamma^{O(S)}\)), and
\item the positions of all work-tape heads (each in \(\{1,\dots,S\}\), or a special value if \(S=0\)).
\end{itemize}
The number of such extended states satisfies
\begin{equation}\label{eq:extstates}
N(S) \;=\; 2^{O(S)}.
\end{equation}
(The constant hidden in \(O(S)\) depends on \(M\): number of tapes, \(|Q|\), \(|\Gamma|\).)

Now fix an input alphabet \(\Sigma\) and consider any string \(x\in\Sigma^*\).
Define the \emph{behavior} of \(x\) at space bound \(S\) to be the function
\(
B_x : \{1,\dots,N(S)\} \to \{1,\dots,N(S)\}
\)
mapping an extended state \(\sigma\) to the extended state \(\sigma'\) obtained as follows:
start \(M\) in extended state \(\sigma\) with the input head positioned \emph{immediately to the right of \(x\)} (at the boundary between \(x\) and the suffix), and run \(M\) until the next time its input head again returns to that boundary position with the same orientation convention (for definiteness: the first time it is at the boundary and about to move to the right).
If such a return happens, let \(\sigma'\) be the extended state then; if it never happens (e.g., the machine halts first), encode that outcome by a designated sink value.
Because \(M\) is deterministic and we restrict attention to computations that stay within \(S\) work cells, this defines a well-defined function \(B_x\) on a finite domain.

The crucial property is \emph{compositionality}: for any \(x,y\),
\begin{equation}\label{eq:composition}
B_{xy} = B_y \circ B_x.
\end{equation}
Intuitively, the effect of first exposing the machine to \(x\) and then to \(y\) is the composition of the two boundary-transfer behaviors.

\subsubsection*{2) Counting behaviors: doubly exponential in \(S\)}
Since \(B_x\) is a function on a set of size \(N(S)\), the number of possible behaviors is at most
\begin{equation}\label{eq:behcount}
\#\mathrm{Beh}(S) \;\le\; N(S)^{N(S)} \;=\; 2^{N(S)\log N(S)} \;=\; 2^{2^{O(S)}},
\end{equation}
using \eqref{eq:extstates}. This is the ``doubly exponential'' bound from the hint.

\subsubsection*{3) Minimal hard input implies all prefixes have distinct behaviors}
Assume there exists an input on which \(M\) uses space \(S>0\).
Let \(w\) be a \emph{shortest} such input, and write \(n=|w|\).
For each prefix \(w_{\le i}\) of length \(i\in\{0,1,\dots,n\}\), consider its behavior \(B_{w_{\le i}}\) at space bound \(S\).

We claim that these \(n+1\) behaviors are all distinct.
Indeed, if \(B_{w_{\le i}} = B_{w_{\le j}}\) for some \(0\le i<j\le n\), write \(w = uvz\) where \(u=w_{\le i}\) and \(uv = w_{\le j}\), so \(v\ne \epsilon\).
By \eqref{eq:composition}, \(B_{uv} = B_v \circ B_u\).
If \(B_{u}=B_{uv}\), then for any continuation \(z\) the machine's boundary interactions with the suffix after \(u\) are identical whether or not \(v\) is present: the composition law forces the same transfer on extended states at the \(u\!\mid\!z\) boundary.
Consequently, the computation of \(M\) on \(uz\) reproduces the same sequence of extended states at the boundary as on \(uvz=w\), and in particular reaches a configuration using \(S\) work cells as well.
Thus \(uz\) is a strictly shorter input requiring space \(S\), contradicting the choice of \(w\).
Hence all prefix behaviors are distinct.

Therefore,
\begin{equation}\label{eq:n-beh}
n+1 \;\le\; \#\mathrm{Beh}(S) \;\le\; 2^{2^{O(S)}}.
\end{equation}

\subsubsection*{4) Inverting the bound and concluding constant space}
From \eqref{eq:n-beh} there exists a constant \(a>0\) (depending only on \(M\)) such that for all \(S\ge 1\),
\begin{equation}\label{eq:invert}
\log\log n \;\le\; aS + a.
\end{equation}
(For example, if \(n \le 2^{2^{aS}}\) then \(\log\log n \le aS\), and additive constants can absorb the ``\(+1\)'' in \(n+1\).)

But on the minimal hard input \(w\) of length \(n\), we also have \(S \le s(n) < c\log\log n\).
Combining with \eqref{eq:invert} gives
\[
S \;<\; c(aS+a) \;=\; (ac)S + ac.
\]
If \(c<1/a\), then \((1-ac)S < ac\), hence
\(
S < \dfrac{ac}{1-ac},
\)
a constant depending only on \(a\) and \(c\).
Thus \(S\) cannot be arbitrarily large: every computation of \(M\) uses at most a fixed constant amount of work space.
Equivalently, \(M\) uses constant space.

\subsubsection*{5) Remark}
This is one of the standard ``lower-than-\(\log\log n\)'' collapse phenomena for space: once the allowed space is below a small enough constant multiple of \(\log\log n\), the machine cannot support unboundedly many distinct prefix behaviors, forcing the space usage to be bounded.
This explains why the language in Problem 2.8.11(g) sits naturally at the \(O(\log\log n)\) threshold.
